\section{\textbf{Question 1: Define replication and the decline effect.
Discuss their
importance.}}\label{question-1-define-replication-and-the-decline-effect.-discuss-their-importance.}

Replication refers to the practice of recreating the conditions of an
experiment (and the steps followed), using different test subjects, in
an effort to confirm the results of previous trials of the experiment.
Called ``the foundation of modern research'' in paragraph three,
replication can be used to confirm a new drug's effectiveness. In the
scientific research process, replication is used to correct the
naturally occurring error amongst researchers of noticing and accepting
results that they hypothesized. Because of this tendency, researchers
may inadvertently ignore/manipulate data to confirm their preconceived
expectations for the outcome of an experiment. Replication serves to
eliminate this ``flaw''; the scientific community can step in and try to
confirm the results of the experiment by replicating it (Paragraph 3).

Joseph Rhine performed an experiment to test the existence of
extrasensory perception, or ESP. In this experiment, one subject showed
the possibility of having ESP. (Paragraph 13). However, when Rhine tried
to replicate the experiment he found that the test subject's supposed
ESP had seemingly vanished. Rhine was the first to call this the
``decline effect''; the disappearance of statistical evidence supporting
a scientific claim (that ESP existed, in Rhine's case). Since Rhine
named this phenomenon, researchers have noticed how consistently the
decline effect occurs. Its reoccurrence suggests a flaw in the
scientific research process; that bias may cause data to be accepted
because it confirms the hypothesis. The decline effect serves as an
important reminder that an experiment never proves truth (Paragraph 38).

\section{\textbf{Question 2: Describe one potential explanation for the
decline
effect.}}\label{question-2-describe-one-potential-explanation-for-the-decline-effect.}

A potential explanation for the decline effect is that initial sample
sizes may be small and/or contain outliers that skew and indicate a
groundbreaking outcome. When the experiment is replicated later, the
total number of subjects sampled increases, which leads to regression
towards the mean (Paragraph 16). However, this doesn't explain why the
results seem to decline in a linear fashion; a fluctuation would be
expected to occur with randomization. Jonathan Schooler's experiments
finding ``verbal overshadowing'' (Paragraph 6) support this regression
towards the mean. Initially, Schooler's experiments indicated a large
amount of ``verbal overshadowing''. After extended replication, Schooler
noticed less and less among his subjects. Schooler eventually concluded
in paragraph 8 that ``his initial batch of research subjects must have
been unusually susceptible to verbal overshadowing,'' or that his
initial batch had outliers.

\section{\textbf{Question 3: Advantages/Disadvantages of an Open-Source
Database}}\label{question-3-advantagesdisadvantages-of-an-open-source-database}

An advantage of researches sharing their data via open-source database
would be that experiment quality would be more readily subject to
evaluation, meaning that overall researchers would increase experiment
quality to maintain a reputable image. A disadvantage of sharing data
this way would be an increase in time spent preparing for an experiment,
only to have the same outcome anyways. Experimenters would need to spend
time uploading intentions and results to this database, but in the end
replication would need to occur to reinforce the original conclusion, no
matter how legitimate the experiment seems.

To combat the decline effect, I would highly recommend increasing sample
size to at least thirty for all experiments, as statistically a sample
size of at least thirty has been proven to show an approximately normal
distribution that accurately reflects the true population distribution.
A large enough sample size will incorporate regression to the mean.
Perhaps by obtaining a statistically significant mean that is close to
the true mean, we will see a decline in the occurrence of the decline
effect.
